\documentclass[review]{elsarticle}

\usepackage{lineno,hyperref}
\usepackage{algorithm}
\usepackage{algpseudocode}
\usepackage{booktabs}
\renewcommand{\algorithmicrequire}{\textbf{Input:}}
\renewcommand{\algorithmicensure}{\textbf{Output:}}
\usepackage{color}
\newcommand{\yl}[1]{\textcolor[rgb]{1.00,0.00,0.00}{#1}}

\modulolinenumbers[5]

\journal{Journal of \LaTeX\ Templates}

%%%%%%%%%%%%%%%%%%%%%%%
%% Elsevier bibliography styles
%%%%%%%%%%%%%%%%%%%%%%%
%% To change the style, put a % in front of the second line of the current style and
%% remove the % from the second line of the style you would like to use.
%%%%%%%%%%%%%%%%%%%%%%%

%% Numbered
%\bibliographystyle{model1-num-names}

%% Numbered without titles
%\bibliographystyle{model1a-num-names}

%% Harvard
%\bibliographystyle{model2-names.bst}\biboptions{authoryear}

%% Vancouver numbered
%\usepackage{numcompress}\bibliographystyle{model3-num-names}

%% Vancouver name/year
%\usepackage{numcompress}\bibliographystyle{model4-names}\biboptions{authoryear}

%% APA style
%\bibliographystyle{model5-names}\biboptions{authoryear}

%% AMA style
%\usepackage{numcompress}\bibliographystyle{model6-num-names}

%% `Elsevier LaTeX' style
\bibliographystyle{elsarticle-num}
% \bibliographystyle{prletters}
%%%%%%%%%%%%%%%%%%%%%%%

\begin{document}

\begin{frontmatter}

\title{Few-shot TAL}
\tnotetext[mytitlenote]{Fully documented templates are available in the elsarticle package on \href{http://www.ctan.org/tex-archive/macros/latex/contrib/elsarticle}{CTAN}.}

%% Group authors per affiliation:
\author{Elsevier\fnref{myfootnote}}
\address{Radarweg 29, Amsterdam}
\fntext[myfootnote]{Since 1880.}

%% or include affiliations in footnotes:
\author[mymainaddress,mysecondaryaddress]{Elsevier Inc}
\ead[url]{www.elsevier.com}

\author[mysecondaryaddress]{Global Customer Service\corref{mycorrespondingauthor}}
\cortext[mycorrespondingauthor]{Corresponding author}
\ead{support@elsevier.com}

\address[mymainaddress]{1600 John F Kennedy Boulevard, Philadelphia}
\address[mysecondaryaddress]{360 Park Avenue South, New York}

\begin{abstract}
This template helps you to create a properly formatted \LaTeX\ manuscript.
\end{abstract}

\begin{keyword}
\texttt{elsarticle.cls}\sep \LaTeX\sep Elsevier \sep template
\MSC[2010] 00-01\sep  99-00
\end{keyword}

\end{frontmatter}

\linenumbers

\section{EXPERIMENTS}
\subsection{Experiment Setup}

\paragraph{Dataset} We evaluate our few-shot action localization algorithm on one popular dataset - THUMOS14 \yl{cite}. THUMOS14 provides temporal annotations for 20 classes and consists of 200 validation and 213 testing videos. The training set contains 2765 trimmed videos from UCF-101 dataset. The few-shot porblem setup requires that the testing classes having no overlap with the training calsses. Following former researches \cite{Yang_2018_CVPR}\cite{Xu2018SimilarityRF}\cite{Xie2021FewShotAL} etc., we will 20 classes are split into 6 classes (THUMOS-val-6) for training and 14 classes (THUMOS-test-14) for testing. 

\paragraph{Metric} We evaluate our madels adopting mean Average Precision (mAP) at different tIoU thresholds (mAP@tIoU). 


\begin{algorithm}
    \caption{Select the closest 50\% instances of each category.}
    \begin{algorithmic}[1]
    \Require Train subset $V^n_i = \{V^n_i\}^{N_1}_{i=1}, V^n_i = \{V^n_i\}^{N_2}_{i=1}, ..., V^n_i = \{V^n_i\}^{N_20}_{i=1}$
    \Ensure Train subset $V^n_i = \{V^n_i\}^{N_1 / 2}_{i=1}, V^n_i = \{V^n_i\}^{N_2/2}_{i=1} ..., V^n_i = \{V^n_i\}^{N_{20}/2}_{i=1}$
    \State Calculating the number to be deleted for each category within each loop: $N^D_i = N_i\times 10\% ,   \quad  i = 1, 2..., 20$
    \For{Loop = 1, 2, 3, 4, 5 do}
        \For{for c = 1, 2, ..., 20 do}
            \State Extract features of $V^n_c $ from train sub-set as $\{f^n_i\}^{N_c}_{i=1} $
            \State  $ \overline{f^{N_c}_c} = \frac{1}{n}\sum\limits_{i=1}^{N_c}\{f^n_i\}^{N_c}_{i=1}\rightarrow feat_c $
            \State  $d^{N_c}_i = \sqrt{\sum\limits_{j=1}^{N_c}(feat_c - f^n_i)^2},\quad i = 1,2..., N_c$ 
            \State Sort $d^{N_c}_i$ in ascending order
            \State Delete $N^D_c$ videos with the largest distance in train subset
        \EndFor
    \EndFor
    \end{algorithmic}
\end{algorithm}

% Table generated by Excel2LaTeX from sheet '类别划分'
\begin{table}[htbp]
  \centering
  \caption{Category division}
    \begin{tabular}{cllll}
    \toprule

          & \multicolumn{1}{c}{group1} & \multicolumn{1}{c}{group2} & \multicolumn{1}{c}{group3} & \multicolumn{1}{c}{group4} \\
    \midrule

          & 1. BaseballPitch & 5. CliffDiving & 11. HammerThrow & 14. LongJump \\
          & 2. BasketballDunk & 6. CricketBowling & 12. HighJump & 15. PoleVault \\
          & 3.  Billiards & 8. Diving & 13. JavelinThrow & 17. SoccerPenalty \\
          & 4. CleanAndJerk & 9. FrisbeeCatch & 16. Shotput & 18. TennisSwing \\
          & 7. CricketShot & 10. GolfSwing & 19. ThrowDiscus & 20. VolleyballSpiking \\

    \bottomrule
    \end{tabular}%
  \label{tab:addlabel}%
\end{table}%



\paragraph{Functionality} The Elsevier article class is based on the standard article class and supports almost all of the functionality of that class. In addition, it features commands and options to format the
$Accuracy =  \frac{TP = TN}{TP + TN + FP + FN}$


\begin{itemize}
\item document style
\item baselineskip
\item front matter
\item keywords and MSC codes
\item theorems, definitions and proofs
\item lables of enumerations
\item citation style and labeling.
\end{itemize}

\begin{algorithm}
    \caption{Divide the 20 categories into 5 categories}
    \begin{algorithmic}[1]
    \Require Train subset $V^n_i = \{V^n_i\}^{N_1/2}_{i=1}, V^n_i = \{V^n_i\}^{N_2/2}_{i=1}, ..., V^n_i = \{V^n_i\}^{N_{20}/2}_{i=1}$
    \Ensure Four groups: $G_1, G_2, G_3, G_4$
    \State Initialize: Extract features of $V_c^n$ from each train sub-set as $\{f_i^n \}_{i=1}^{N_c}$
    \State $ \overline{f^{N_c}_c} = \frac{1}{n}\sum\limits_{i=1}^{N_c}\{f^n_i\}^{N_c}_{i=1}\rightarrow feat_c $
    \State Compute the Euclidean between each $fear_c$ as $d_{c\times n}$
    \For {k = 1, 2, 3, 4 }
	    \State $group \leftarrow G_1, G_2, G_3, G_4$
    	\While{$length(G_k) \le 5$}
		    \State Find the minimum value in the $d_{c\times n}$ matrix, horizontal as x and vertical as y
		    \State $d[x][y] \leftarrow \infty$
		    \If{$x \notin group$ and $x \le 20 $ }
			    \State $G_k \leftarrow x$
		    \EndIf
            \If{$y \notin group$ and $y \le 20$ and $y \neq x$}
			    \State $G_k \leftarrow y$
		    \EndIf
            \If{$G_k$ have changed}
		    	\State $ feat_{c+1} = feat_c + \frac{1}{length(G_k)}\sum \limits_{g\in   G_k}feat_g$
         	\EndIf
        \EndWhile
    \EndFor

    \end{algorithmic}
\end{algorithm}


\section{Front matter}

The author names and affiliations could be formatted in two ways:
\begin{enumerate}[(1)]
\item Group the authors per affiliation.
\item Use footnotes to indicate the affiliations.
\end{enumerate}
See the front matter of this document for examples. You are recommended to conform your choice to the journal you are submitting to.

\section{Bibliography styles}

There are various bibliography styles available. You can select the style of your choice in the preamble of this document. These styles are Elsevier styles based on standard styles like Harvard and Vancouver. Please use Bib\TeX\ to generate your bibliography and include DOIs whenever available.

Here are two sample references: \cite{Feynman1963118,Dirac1953888}.

\section*{References}
\bibliography{mybibfile}


\end{document}